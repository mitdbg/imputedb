\documentclass{beamer}

\usepackage[utf8]{inputenc}
\usepackage[T1]{fontenc}
\usepackage[english]{babel}
\usepackage[activate={true,nocompatibility},final,kerning=true,spacing=true,factor=1100,stretch=10,shrink=10]{microtype}
\usepackage{amsfonts,amssymb,amsmath}
\usepackage{xcolor}
\usepackage{algorithm}
\usepackage[noend]{algpseudocode}
\usepackage[style=ieee]{biblatex}
\usepackage{booktabs}
\usepackage{hyperref} % load after other packages
\usepackage{todonotes}
\presetkeys{todonotes}{inline}{}


\addbibresource{references.bib}

\makeatletter
\newcommand*{\centerfloat}{%
  \parindent \z@
  \leftskip \z@ \@plus 1fil \@minus \textwidth
  \rightskip\leftskip
  \parfillskip \z@skip}
\makeatother

\DeclareMathOperator*{\argmax}{arg\,max}
\DeclareMathOperator*{\argmin}{arg\,min}

\MakeRobust{\Call}

\renewcommand{\emptyset}{\varnothing}

\newcommand{\ProjectName}{ImputeDB}



\title{\ProjectName{}: A Database for Missing Data}
\author{Jose Cambronero \and Micah Smith \and John K. Feser}
\institute{MIT}
\date{\today{}}

\begin{document}
\frame{\titlepage}

\begin{frame}{Overview}
\begin{itemize}
	\item Handling incorrect or dirty data is a complex and challenging problem for data scientists
	\item $\text{missing data} \subset \text{dirty data}$
	\item Cost of running imputation on entire dataset
	\item Lack of flexibility
	\item \ProjectName{} allows users to interact with dataset as though it were clean.
	\item User should never see missing data or have to modify query.
\end{itemize}
\end{frame}

\begin{frame}[fragile]{Motivating Example}
\todo{motivating example}
\end{frame}

\begin{frame}[fragile]{Query Planning: Overview}
\begin{itemize}
	\item $\mu_{min}, \mu_{max}, \delta$: new relational algebra operators
	\item Cost of computation vs. information loss: key tradeoff controlled by $\alpha$ parameter
	\item Placement of operators and dirty sets generated considered in planning.
	\item Histogram transformations used to provide up-to-date cardinality and missing value estimates during planning.
	\item Dynamic programming approach, extends System R optimization\cite{blasgen1981system}
	\item Exponential complexity, but tractable in practice with low planning times.
\end{itemize}
\end{frame}

\begin{frame}[fragile]{Query Planning: Top-Level Algorithm}

\begin{algorithm}[H]
 \begin{algorithmic}
   \scriptsize
   \Require{set of tables $T$, filter predicates $\Phi$, join predicates $\Psi$, a set of projection attributes $P$, optional set of grouping attributes $G$ and aggregator function $A$}
   \Ensure{Returns optimized query plan.}
   
  \Function{Plan}{$T, \Phi, \Psi, P, G, A$}
 \State  $Q \gets \{ \}$ \Comment{map from table and dirty set to plans }
   \For{$t \in T$}
   	\State $\phi \gets \Call{GetFilters}{t, \Phi}$
	\State $P_{t} \gets \Call{OptFilter}{t, \phi}$
	\State $Q \gets Q \cup \{ (t, \Call{Dirty}{q}) \mapsto q | q \in P_t \}$
   \EndFor
   
   \State $M \gets \{ \}$ \Comment{map from dirty set to plans, contains all joins necessary}
   \State $M \gets  \Call{OptJoin}{Q, \Psi}$	\Comment{optimize joins}
   \If{$G \neq \emptyset \land A \neq NULL$}
   	\State $attrs \gets \Call{Attrs}{G} \cup \Call{Attrs}{A}$
   	\State $M \gets \Call{OptRel}{\bigcup_{q \in M} \Call{GroupBy}{\Call{AddImpute}{q, attrs}, G, A}}$
   \Else
  	\State $M \gets \Call{OptRel}{\bigcup_{q \in M} \pi_{P}(\Call{AddImpute}{q, P}) }$
  \EndIf 
   
\Return $\argmin_{q \in M} Cost(q)$
\EndFunction
  
  \end{algorithmic}
  \caption{Top-level query planner with imputations.}
\label{algo:top-level-planner}
\end{algorithm}

\end{frame}

\begin{frame}[fragile]{Imputation Algorithm}
\begin{itemize}
	\item Iterative algorithm using regression trees: \textit{Chained Equations Regression Trees} (CERT)\cite{burgette2010multiple}
	\item Significant cost to perform on entire dataset
	\item \ProjectName{} allows flexible use
\end{itemize}

\begin{algorithm}[H]
  \scriptsize
  \begin{algorithmic}
    \Require{$T$ is a table. $D$ is a set of $T$ attributes that need to be imputed, and $C$ is a set of $T$ attributes that has complete data }
    \Ensure{Returns an imputed $T$}
    
    \Function{ImputeWithCERT}{$H, op$}
    	\State $T' \gets \Call{ImputeRandom}{T, D}$
	\For{$n\ EPOCHS$}
		\For{$d \in D$}
			\State $imp \gets \Call{TrainRT}{\pi_C(T'), \pi_d{T'}}$
			\State $T' \gets (\pi_C(T'), \Call{PredictRT}{imp, T})$
		\EndFor
	\EndFor
	\Return{T'}
	\EndFunction
  \end{algorithmic}
  \caption{An algorithm for chained imputation using regression trees}
  \label{algo:imputation-strategy}
\end{algorithm}
\end{frame}

\begin{frame}[fragile]{Experiments: Data}
\begin{itemize}
	\item Real dataset: American Community Survey (U.S. Census Bureau)
		\begin{itemize}
			\item Preprocessed version courtesy of authors of~\cite{akande2015empirical}.
			\item 671,153 rows and 37 columns, all integers.
			\item Randomly deleted 10\% of fields to create dirtied variant
			\item Can write typical data analyst queries involving filters and aggregates
		\end{itemize}
	\item Synthetic dataset: drawn from uniform distribution $[0,100)$
		\begin{itemize}
			\item10,000 rows and 10 columns
			\item Randomly deleted 30\% ti create dirtied variant
			\item Crafted ad-hoc queries using joins
		\end{itemize}
\end{itemize}
\end{frame}

\begin{frame}[fragile]{Experiments: Queries}
\begin{figure}
  \centerfloat
  \tiny
  \begin{tabular}{ll}
\toprule
\# & Query \\
\midrule
0 & \verb|SELECT BLD FROM acs;| \label{q0} \\
1 & \verb|SELECT BLD as units_in_structure, COUNT(ST) as estimate FROM acs GROUP BY BLD;| \label{q1} \\
2 & \verb|SELECT BATH as has_bath, COUNT(ST) as ct FROM acs GROUP BY BATH;| \label{q2} \\
3 & \verb|SELECT ACR as lotsize, AVG(BDSP) as avg_num_bedrooms FROM acs GROUP BY ACR;| \label{q3} \\
4 & \verb|SELECT PSF as has_sub_families, SUM(NP) as num_people FROM acs GROUP BY PSF;| \label{q4} \\
5 & \verb|SELECT AVG(NP) as avg_num_people FROM acs;| \label{q5} \\
6 & \verb|SELECT BDSP as num_bedrooms, AVG(ACR) as avg_lot_size FROM acs WHERE VEH >= 2 GROUP BY BDSP;| \label{q6} \\
\bottomrule
\end{tabular}

  \caption{The queries used in our experiments.}
  \label{fig:queries}
\end{figure}
\end{frame}

\begin{frame}[fragile]{Experiments: Results}
\begin{figure}
  \tiny
  \centerfloat
  \begin{tabular}{cSSSSSSS}
\toprule
\multicolumn{2}{c}{} & \multicolumn{2}{c}{Imputed ($\alpha=0.0$)} & \multicolumn{2}{c}{Imputed ($\alpha=1.0$)} \\
\cmidrule(r){3-4}
\cmidrule(l){5-6}
\# & \multicolumn{1}{c}{Base error} & \multicolumn{1}{c}{Error} & \multicolumn{1}{c}{Time (s)} & \multicolumn{1}{c}{Error} & \multicolumn{1}{c}{Time (s)} \\
\midrule
\ref{q1} & \multicolumn{1}{c}{--} & \multicolumn{1}{c}{--} & 1 & \multicolumn{1}{c}{--} & 17 \\
\ref{q2} & 1.66e+04 & 0.00e+00 & 1 & -1.14e+04 & 9 \\
\ref{q3} & 4.78e+04 & 0.00e+00 & 3 & -4.77e+04 & 9 \\
\ref{q4} & 1.86e-03 & 0.00e+00 & 2 & 5.07e-02 & 32 \\
\ref{q5} & 2.52e+05 & 0.00e+00 & 1 & -2.19e+05 & 25 \\
\ref{q6} & 9.89e-04 & 0.00e+00 & 1 & 9.89e-02 & 8 \\
\ref{q7} & 3.76e-02 & 0.00e+00 & 2 & 4.21e-02 & 55 \\
\ref{q8} & 0.00e+00 & 0.00e+00 & 2 & 0.00e+00 & 3 \\
\ref{q9} & \multicolumn{1}{c}{--} & \multicolumn{1}{c}{--} & 0 & \multicolumn{1}{c}{--} & 7 \\
\ref{q10} & \multicolumn{1}{c}{--} & \multicolumn{1}{c}{--} & 0 & \multicolumn{1}{c}{--} & 0 \\
\ref{q11} & 0.00e+00 & 0.00e+00 & 0 & 1.00e-01 & 0 \\
\ref{q12} & \multicolumn{1}{c}{--} & \multicolumn{1}{c}{--} & 0 & \multicolumn{1}{c}{--} & 0 \\
\ref{q13} & 0.00e+00 & 0.00e+00 & 0 & 3.81e+01 & 72 \\
\bottomrule
\end{tabular}

  \caption{Running time and error for queries with different imputation levels.}
  \label{fig:experiments}
\end{figure}
\end{frame}



\begin{frame}[fragile]{Related work}
\begin{itemize}
	\item Long history of imputation in statistical learning community.
		\begin{itemize}
			\item Akande et al (\cite{akande2015empirical}) explored ACS data to compare imputation strategies.
			\item Burgette and Reiter (\cite{burgette2010multiple}) introduce sequences of regression trees.
		\end{itemize}
	\item Long history of null value semantics in database community.
		\begin{itemize}
			\item Treatment of nulls described as early as  \cite{codd1973understanding}
			\item Statistical databases, such as BayesDB (\cite{mansinghka2015bayesdb}) combine statistical inference techniques and traditional querying.
		\end{itemize}
	\item \ProjectName{} first to integrate imputation into query planning with cost-based optimizer.
	\item \ProjectName{} treatment of nulls completely abstracted from user perspective: SQL logical independence.
\end{itemize}
\end{frame}


\begin{frame}[fragile]{Conclusion and Future work}
\begin{itemize}
	\item Missing values and their imputation can successfully be integrated into the relational calculus
	\item By taking a dynamic programming approach, we can consider a variety of operator placements
and input columns, while keeping planning tractable in real-world examples.
	\item Considered series of real-world and fabricated queries, showing viability of approach.
	\item Future work:
		\begin{itemize}
			\item Extend imputation operators with global information (\textit{impute by need})
			\item Introduce multiple imputation algorithms into optimizer for consideration in plans \
			\item Implement in a production-quality database.
		\end{itemize}
\end{itemize}
\end{frame}

\end{document}
