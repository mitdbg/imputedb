\documentclass[preprint]{vldb}
\usepackage{balance}
\usepackage[utf8]{inputenc}
\usepackage[T1]{fontenc}
\usepackage[english]{babel}
\usepackage[activate={true,nocompatibility},final,kerning=true,spacing=true,factor=1100,stretch=10,shrink=10]{microtype}
\usepackage[subtle]{savetrees}
\usepackage{amsfonts,amssymb,amsmath}
\usepackage{xcolor}
\usepackage{algorithm}
\usepackage[noend]{algpseudocode}
\usepackage[style=savetrees,firstinits=true,maxcitenames=2]{biblatex}
\usepackage{booktabs}
\usepackage{siunitx}
\usepackage{graphicx} % includegraphics
\usepackage{listings} % code samples with syntax hl
\usepackage{stmaryrd}
\usepackage{qtree} % for simple tree diagrams
\usepackage{caption}
\usepackage{subcaption}
\usepackage{xcolor}
\usepackage{tikz}
\usetikzlibrary{positioning}
\usepackage{xspace}
\usepackage[all]{nowidow}
\usepackage{enumitem}
\usepackage{hyperref} % load after other packages
\usepackage{cleveref}

\newcommand{\srm}[1]{\textcolor{red}{SAM:#1}}
\addbibresource{references.bib}

\renewcommand{\algorithmicrequire}{\textbf{Input:}}
\renewcommand{\algorithmicensure}{\textbf{Output:}}

\hyphenation{Im-pute-DB}

% listings settings
\lstset{ %
  backgroundcolor=\color{white},   % choose the background color; you must add \usepackage{color} or \usepackage{xcolor}; should come as last argument
  basicstyle=\ttfamily\footnotesize,        % the size of the fonts that are used for the code
  breakatwhitespace=true,         % sets if automatic breaks should only happen at whitespace
  breaklines=true,                 % sets automatic line breaking
  captionpos=b,                    % sets the caption-position to bottom
  commentstyle=\color{red},    % comment style
  %frame=single,	                   % adds a frame around the code
  keepspaces=true,                 % keeps spaces in text, useful for keeping indentation of code (possibly needs columns=flexible)
  keywordstyle=\color{blue},       % keyword style
%  numbers=left,                    % where to put the line-numbers; possible values are (none, left, right)
  numbersep=5pt,                   % how far the line-numbers are from the code
%  numberstyle=\tiny\color{gray}, % the style that is used for the line-numbers
  rulecolor=\color{black},         % if not set, the frame-color may be changed on line-breaks within not-black text (e.g. comments (green here))
%  stepnumber=2,                    % the step between two line-numbers. If it's 1, each line will be numbered
  stringstyle=\color{red},     % string literal style
  tabsize=2,	                   % sets default tabsize to 2 spaces
  language=SQL
}

\makeatletter
\newcommand*{\centerfloat}{%
  \parindent \z@
  \leftskip \z@ \@plus 1fil \@minus \textwidth
  \rightskip\leftskip
  \parfillskip \z@skip}
\makeatother

\DeclareMathOperator*{\argmax}{arg\,max}
\DeclareMathOperator*{\argmin}{arg\,min}

\MakeRobust{\Call}

\renewcommand{\emptyset}{\varnothing}

\newcommand{\ProjectName}{ImputeDB\xspace}
\newcommand{\sys}{\ProjectName}
\newcommand{\doms}[2]{#1 \succ #2}

% definitions/theorems etc
\newtheorem{definition}{Definition}
\newtheorem{theorem}{Theorem}
\newtheorem{case}{Case}

% new section names for cref
\Crefname{query}{Query}{Query}

\title{Query Optimization for Dynamic Imputation}

\author{
  Jos\'e Cambronero\thanks{Author contributed equally to this paper.} \\
  MIT CSAIL \\
  \texttt{jcamsan@csail.mit.edu}
  \and
  John K. Feser\footnotemark[1] \\
  MIT CSAIL \\
  \texttt{feser@csail.mit.edu}
  \and
  Micah J. Smith\footnotemark[1] \\
  MIT LIDS \\
  \texttt{micahs@mit.edu}
  \and
  Samuel Madden \\
  MIT CSAIL \\
  \texttt{madden@csail.mit.edu}}

\newcommand{\fix}{\marginpar{FIX}}
\newcommand{\new}{\marginpar{NEW}}
\newcommand{\todo}[1]{{\color{red}\textbf{TODO:} #1}}
\newcommand{\todobox}[2]{{\color{red}\fbox{\parbox{\columnwidth}{\textbf{TODO:} \textit{#1}\\#2}}}}
\newcommand{\review}[1]{{\color{blue}#1}}

% results that can be used in any section
\newcommand{\demorows}{10175}
\newcommand{\labexrows}{9813}
\newcommand{\acsbaseresultminutes}{355 minutes}
\newcommand{\acsbaseresulthours}{just under 6 hours}
\newcommand{\acsimputedbzeroresult}{4 seconds}
\newcommand{\acsimputedboneresult}{1 second}
\newcommand{\lowxalphazero}{10} % the smallest speed up, comparing a=0 to base table imputation
\newcommand{\highxalphazero}{140} % the largest speed up, comparing a=0 to base table imputation 
\newcommand{\highxalphaone}{1400} % the largest speed up, comparing a=0 OR a=1 to base table imputation 
\newcommand{\lowsmapealphazero}{0} % the smallest SMAPE difference, comparing a=0 to base table imputation
\newcommand{\highsmapealphazero}{6} % the largest SMAPE difference, comparing a=0 to base table imputation
\newcommand{\highsmapealphaone}{20} % the largest SMAPE difference, comparing a=1 to base table imputation
%\newcommand{\highxalphaone}{16000}
\newcommand{\nullv}{NULL}

\newcommand{\runtimetreelowzero}{75 ms}
\newcommand{\runtimetreehighzero}{1 second}
\newcommand{\runtimetreelowone}{12 ms}
\newcommand{\runtimetreehighone}{19 ms}
\newcommand{\runtimetreelowbase}{6.5 seconds}
\newcommand{\runtimetreehighbase}{27 seconds}

\newcommand{\planningtimelow}{0.7 ms}
\newcommand{\planningtimelowpct}{0}
\newcommand{\planningtimehigh}{2.8 ms}
\newcommand{\planningtimehighpct}{14}

\newcommand{\planningtimemax}{113 ms} % max planning time of any single query
\newcommand{\planningtimepninetynine}{4 ms} % 99 percentile planning time of any single query
\newcommand{\planningtimepninetyninepointnine}{14 ms} % 99.9 percentile planning time of any single query

\newcommand{\runtimemeanlow}{12 ms}
\newcommand{\runtimemeanhigh}{25 ms}
\newcommand{\runtimemeanlowbase}{24 ms}
\newcommand{\runtimemeanhighbase}{44 ms}
\newcommand{\runtimehotdecklow}{12 ms}
\newcommand{\runtimehotdeckhigh}{29 ms}
\newcommand{\runtimehotdecklowbase}{21 ms}
\newcommand{\runtimehotdeckhighbase}{92 ms}

\newcommand{\smapehotdecklow}{0}
\newcommand{\smapehotdeckhigh}{3}
\newcommand{\smapehotdeckhighoutlier}{24}

\begin{document}

\maketitle

\begin{abstract}
  Missing values are common in data analysis and present a usability challenge.
  Users are forced to pick between removing tuples with missing values or creating a cleaned version of their data by applying a relatively expensive imputation strategy.
  Our system, \ProjectName{}, incorporates imputation into a cost-based query optimizer, performing necessary imputations on-the-fly for each query.
  This allows users to immediately explore their data, while the system picks the optimal placement of imputation operations.
  % This placement is based on the specific query issued and a user-provided parameter that trades off data quality and runtime performance.
  % We incorporate the imputation operators into the relational algebra, allowing for traditional optimizations to take place.
  We evaluate this approach on three real-world survey-based datasets.
  % from the \textit{Center for Disease Control and Prevention} and \textit{freeCodeCamp}.
  Our experiments show that our query plans execute between \lowxalphazero{} and \highxalphazero{} times faster than first imputing the base tables.
  Furthermore, we show that the query results from on-the-fly imputation differ from the
  traditional base-table imputation approach by \lowsmapealphazero{}--\highsmapealphaone{}\%, in one error measure.
  Finally, we show that while dropping tuples with missing values that fail query constraints discards 6--78\% of the data, on-the-fly imputation loses only 0--21\%.
\end{abstract}

\section{Introduction}

% Context
Handling incorrect or dirty data is a complex and challenging problem for analysts\todo{add significant citations here}.
% Problem
One way in which a dataset can be dirty is for parts of it to be missing altogether.
Missing data, if handled naively, can cause analyses to be incorrect.
According to some estimates, data scientists spend between 50 and 80 percent of their time performing tasks such as collecting and cleaning datasets~\cite{data-science-cleaning}.

% Solution
To handle this problem, users may manually clean their dataset by performing some statistical analysis to replace missing data elements with likely values.
This process is called \emph{imputation}.
We have developed a system called \ProjectName{}, which is the first database to use a query optimization algorithm which allows imputation to occur on the fly, during query execution. \emph{The key insight behind \ProjectName{} is that imputation
only needs to be performed on the data relevant to a particular query and
that this subset is generally much smaller than the entire database.} 

% Background
While traditional imputation methods work over the entire data set and replace all missing values, running a sophisticated imputation algorithm over a large data set can be very expensive.
A simpler approach might drop all rows with missing values, but not only can this introduce bias into the results, but this may result in discarding all of the data.
Finally, for many queries, it is not necessary to clean the entire data set, and a significant speed up can be obtained by cleaning only the necessary parts.
In contrast to existing systems, \ProjectName{} exploits this observation and avoids imputing over the entire data set, as done with traditional methods. \ProjectName{}
carefully plans the placement of different imputation operations throughout the query plan, resulting in a 
significant speedup in query execution. The focus on of our work is not on the imputation algorithm but rather on the planning involved in its use in non-trivial query plans. 

\todobox{find appropriate location for this}
{In a previous project, she
worked with the American Community Survey data (671,153 rows), and performing imputation on the whole
dataset took around 75 minutes\todo{Here, we can reference the number from our own experiment (done), or the number from Akande et al: ``performing imputation on just a 1.5 percent random sample took 45 minutes''}.}

% Prior work
There is relevant prior work in three broad areas: statistics, databases, and time-series forecasting.
Prior work in statistics\todo{add more specific subarea qualifier} has focused on imputation quality~\cite{burgette2010multiple} and runtime cost~\cite{akande2015empirical}.
Prior work in databases has a established simple semantics for handling tuples with missing values~\cite{codd1973understanding,grant1977null}.
Prior work in time-series forecasting has incorporated forecasting extensions into domain-specific databases and characterized their behavior~\cite{parisi2011embedding,parisi2013temporal,duan2007processing}.
\ProjectName{} contributes the first optimizer to plan data imputation on a per query basis, allowing users to use standard SQL on real-world data sets with missing values
and express preferences in quality of data imputation versus running time. The guiding design principle behind \ProjectName{} is that the user should never see missing data or have to modify their queries to account for it.

% Implications
Our approach enables an exploratory analysis workflow in which the analyst can issue standard SQL queries over a data set, even if that data has missing values, and get
queries that execute in a fraction of the time\todo{add actual performance figures here} it takes to impute base tables and then run queries (the traditional approach). Furthermore, the results obtained with this approach
are similar in quality to those obtained with the base-table imputation approach, as indicated by low\todo{add actual value here} Symmetric-Mean-Absolute-Percentage-Errors in our empirical results
over real-world data sets (see~\Cref{sec:experiments} for details).

% Scope
\ProjectName{} is designed to enable early data exploration, by allowing analysts to run their queries without explicitly dealing with imputation.
However, we optimize query plans with respect to a heuristic measure of imputation quality, and provide no hard guarantees about the accuracy of the results.

\subsection{Contributions}
This paper makes the following contributions:
\begin{itemize}
\item \textbf{Relational algebra with imputation:}
  We extend the standard relational algebra with two new operators to represent imputation operations: impute ($\mu$) and drop ($\delta$).
  Impute operation fills in missing data values using a regression model (CE-CART)~\cite{burgette2010multiple}.
  Drop operation simply drops tuples which contain null values.
\item \textbf{Model of imputation quality and cost:}
  We extend the traditional cost model for query plans to incorporate a measure of the quality of the imputations performed.
  We use the cost model to abstract over the imputation algorithm used.
  To add an additional imputation technique, it is sufficient to characterize it with two functions: one to describe its running time and one to describe the quality of its results.
\item \textbf{Query planning with imputation:}
  We present the first query planning algorithm to jointly optimize for running time and the quality of the imputed results.
  It does so by maintaining multiple sets of Pareto optimal plans according to the cost model.
  By deferring selection of the final plan, we make it possible for the user to trade off running time and result quality.
\item \textbf{Proof of optimality:}
 We prove that our cost-model and planning algorithm produces a sound and complete final Pareto frontier, making the final plan choice optimal under our search space and the user's preferences.
\end{itemize}

\ProjectName{} leverages the combined contributions to plan fast queries on data with missing values, while producing comparable results to traditional approaches.
This provides the opportunity to increase productivity for users who frequently encounter missing values in their data.

% Summary of related work

% Although this manual imputation solves the problem of missing data, in the age of big data it may be very expensive to run an imputation algorithm on an entire dataset.
% Additionally, it may not be necessary to completely clean the data to make it usable.
% Some users may be willing to run queries on dirty data, simply ignoring any missing values, as long as they do not have to pay the cost of imputation.
% Others may want to run queries on a subset of the data, and so do not need to impute
% every field in every record. Yet others may want to customize the 
% imputation algorithm for the tradeoffs and demands of a particular domain.

% In this paper, we present \ProjectName{}, a database system which is designed to interact with a dirty dataset as though it were clean.
% To achieve this goal, we perform imputation on the fly, during query execution.
% Performing imputation at query time allows our system to impute only the data necessary to run the query, and it allows users to flexibly trade imputation quality for computation time.

%%% Local Variables:
%%% mode: latex
%%% TeX-master: "main"
%%% End:

\section{Examples}

Consider the case of an analyst looking to explore the sources and causes of
polling error in a presidential election. One hypothesis \cite{zukin2015s}
is that the rapid switch from landlines to cellphones has increased the
difficulty of contacting potential poll respondents. The analyst may
begin her analysis by comparing the polling error in a state with the
distribution of households with landlines, for the set of states for which
polling error was most pronounced. In order to probe deeply, the analyst
accesses a household-level survey of household characteristics that includes
whether the household has a landline. One such survey --- which will be
discussed further (Sec~\ref{subsec:datasets}) --- is the American Community Survey (ACS),
conducted by the US Census Bureau.

Surveys are rife with missing values due to non-response and other issues, and
the ACS is no exception. In this case, a non-negligible fraction of respondents omitted
information on whether they have a landline. The analyst hypothesizes that older houses are
more likely to have landlines and that households that are relatively new are less likely to
have landlines. 

The analyst could take advantage of these presumed correlations to impute the missing values
in an expensive operation\footnote{One would not necessarily want to impute the entire
database once and store the imputed values, as in many settings the imputation strategy
would be targeted for a specific question or domain.}. On the other hand, the query
could be run directly and rows with missing values would be dropped entirely. With
ImputeDB, the analyst could impute the relevant subset of data on-the-fly, achieving
better performance without losing the information in the dropped examples.

The analyst submits a query to ImputeDB which finds an optimized query plan. The resulting
query imputes missing values for \verb!acs.TEL! after states with high polling error have
been selected.

\begin{verbatim}
SELECT polling.ST, AVG(acs.TEL)
FROM polling, acs
WHERE polling.ST = acs.ST
  AND polling.ERROR > 50         -- 5 percentage points
GROUP BY polling.ST
\end{verbatim}

\todo{Explore a couple of plans generated, using dot visualizer. Confirm that the actual
plans match my guess in last sentence.}

%%% Local Variables:
%%% mode: latex
%%% TeX-master: "main"
%%% End:

\section{Algorithm}

At a high level, our goal is to select a query plan that minimizes a metric which combines the cost of the query and the quality of the results.
Precisely, we will find a query plan $Q$ that minimizes $\text{Cost}(Q) = \alpha \times \text{Time}(Q) + (1 - \alpha) \times \text{Loss}(Q)$. $\alpha$ is a parameter to the query optimizer that controls the emphasis on quality versus performance. $\alpha = 1.0$ means that the query should be as fast as possible, $\alpha=0.0$ means that the query should be as accurate as possible.

\subsection{Search space}
To reduce the size of the query plan search space, only plans that fit the following template are considered. First, all filters are pushed to the leaves of the query tree, immediately after the scans. Joins are performed after filtering, and only left-deep plans are considered. Any group-by/aggregate will be performed after the final join. Finally, projections are placed at the root of the query tree. The space of query plans is similar to that considered by Selinger \todo{Selinger cite}, with the addition of imputation operators.

\subsection{Imputation operators}
We introduce two new relational operators to perform imputation: impute ($\mu$) and drop ($\delta$). Each operator takes arguments $(C, R)$ where $C$ is a set of attributes and $R$ is a relation. Impute uses a machine learning method to replace all null values with non-null values for attributes in $C$ in the relation $R$. Drop simply removes all tuples which have a null value for some attribute in $C$ from $R$. Both operators guarantee that the resulting relation will contain no null values for attributes in $C$. 

\subsection{Imputation placement}
\label{sec:placement}
Imputation operators must be placed so that no relational operator receives a tuple containing null in an attribute that the operator examines, regardless of the state of the data in the base tables.

Imputation operators can be placed at any point in the query plan, but to meet the guarantee that no non-imputation operator sees a null value, there are cases where an imputation operator is required. To track these cases, each query plan is associated with a set of dirty attributes $D$. An attribute $c$ is \emph{dirty} in some relation if the values for $c$ contain null. We compute a dirty set for each base table using the table statistics, which track the number of null values in each column. If we apply an imputation operator to a relation $R$ with a dirty set $D$, $\mu_C (R)$ or $\delta_C (R)$, the resulting query has a dirty set $D' = D \ C$.  Applying a projection $\pi_C(R)$ produces a dirty set $D' = D \cap C$. A join $R_1 \Join_\psi R_2$ with dirty sets $D_1$ and $D_2$  produces a dirty set $D' = D_1 \cup D_2$.  Filters do not change the dirty set. 

The dirty set over-approximates the set of attributes that contain null. For example, a filter might remove all tuples which contain null without changing the dirty set, forcing an unnecessary imputation. We choose to over-approximate the dirty set to avoid the possibility of dropping a tuple that contains a null value without explicitly imputing the value or applying a drop operator.

\subsection{Query planning}
The input to our query planner is a tuple $(T, \Phi, \Psi, P, G, A)$: a set of tables $T$, a set of filter predicates $\phi_t, t \in T$, a set of join predicates $\psi_(t_1, t_2), t_1, t_2 \in T$, a set of attributes $P$, and an optional set of attributes $G$ and aggregator function $A \in \{\text{Max}, \text{Min}, \text{Sum}, \text{Avg}, \text{Count}\}$.

The query planner must select a join ordering in addition to placing imputation operators as described in Section~\ref{sec:placement}.

To reduce the search space, we only consider the minimal imputation, the maximal imputation, and the minimal drop. The minimal imputation (resp. drop) only imputes (resp. drops) the columns required by the relational operator immediately following the imputation. The maximal imputation imputes all columns in the relation, regardless of which are required. 

\begin{algorithm}
  \begin{algorithmic}
    \Require{$q$ is a query plan, $C_{req}$ is a set of attributes that must be imputed.}
    \Ensure{Returns a set of query plans such that $\Call{Dirty}{q'} \cap C_{req} = \emptyset$.}
    \Function{AddImpute}{$q, C_{req}$}
    \State $C_{min} \gets \Call{Dirty}{q} \cap C_{req}$
    \If{$C_{min} = \emptyset$}
    \State \Return $\{q\}$
    \Else
    \State \Return $\{\mu_{\Call{Dirty}{q}}(q), \mu_{C_{min}}(q), \delta_{C_{min}}(q)\}$
    \EndIf
    \EndFunction
    
    \State

    \Require{$Q$ is a set of query plans.}
    \Ensure{Returns an optimal query plan for each distinct dirty set in $Q$.}
    \Function{OptRel}{$Q$}
    \State $D \gets \{\Call{Dirty}{q} ~|~ q \in Q\}$
    \State \Return $\{\argmin_{q \in Q \land \Call{Dirty}{q} = d} \Call{Cost}{q} ~|~ d \in D\}$
    \EndFunction

    \State

    \Require{$t$ is a table and $\phi$ is a filter predicate.}
    \Ensure{Returns a set of optimal query plans for scanning and filtering $t$, with distinct dirty sets.}
    \Function{OptFilter}{$t, \phi$}
    \State \Return $\Call{OptRel}{\Call{AddImpute}{t, \Call{Attrs}{\phi}}}$
    \EndFunction

    \State

    \Require{$Q$ is a set of query plans and $\Psi$ relates query plans with join predicates.}
    \Function{OptJoin}{$Q, \Phi$}
    \State $Q \gets \emptyset$
    \For{$q \in Q$}
    
    \EndFor
    \EndFunction
  \end{algorithmic}
  \caption{An algorithm for query planning with imputations.}
\end{algorithm}

\begin{verbatim}		
# For a set of tables (with filter predicates) and join predicates, return the optimal plans.
OptJoin(T⁆, join_preds):
	Q = empty
	for t, pred in T:
		S = T \ {t, pred}
		# Get the optimal plans for the left and right arguments to the join.
		for r_l, r_r in OptJoin(S) x OptFilter(t, pred):
			if r_l, r_r in join_preds:
				# Add imputation to the left and right arguments.
				Q = Q union OptRel(Join(r_l', r_r') for r_l' in AddImpute(r_l, Attrs(join_pred)), r_r' in AddImpute(r_r, Attrs(join_pred)))
	return OptRel(Q)
# Extended cost calculation to account for information loss/computation cost in imputations
Cost(q):
  case drop_min(q', d) =>  info_loss(drop, q', attrs(q') intersect d) + Cost(q')
  case impute_all(q') => info_loss(impute, q', attr(q')) + impute_cost(q') + Cost(q')
  case impute_min(q', d) => info_loss(impute, q', attrs(q') intersect d) + impute_cost(q'[d]) + Cost(q')
  case _ => normal cost calculation

  
info_loss(op, t, a):
    # beta: some scaling factor
    case info_loss(drop, t, a) => beta * sum(for ct_missing(t.a) in a)
    case info_loss(impute, t, a) => beta *sum(for ct_missing(t.a) / ct_complete(t) in a)

impute_cost(t):
  # alpha: some scaling factor
  alpha * (ntuples(t) + ct(attr(t)))
\end{verbatim}

\subsection{Histogram computation}
Histogram adjustment algorithm:
Invariant: the distribution of non-null values is preserved
Drop(t, ix) => remove nulls from t table statistics for indices ix
Impute(t, ix) => redistribute nulls from t table statistics for indices ix to other buckets based on existing distribution
Filter(t, predicate) => scale all buckets of t table statistics and null counts by predicate selectivity
Join(t1, t2) => estimate cardinality of join result (following selinger, if non-equality, then 30% of product,
If equality and no primary/foreign key relationship then the maximum of the two cardinalities, otherwise
If equality and primary/foreign key relationship then the size ofthe foreign key relationship table). Scale t1's statistics and null counts to total the new cardinality estimate. Do same for t2. Union histograms.

\subsection{Complexity}

%%% Local Variables:
%%% mode: latex
%%% TeX-master: "main"
%%% End:

\review{
\section{Implementation}\label{sec:implementation}
Adding imputation and our optimization algorithm to a standard SQL database requires modifications to several key database components.
In this section we discuss the changes required to implement dynamic imputation. We based this directly on our experience implementing
\ProjectName{} on top of SimpleDB~\cite{simpledb}, a teaching database used at MIT, University of Washington, and Northwestern University, among others.

\begin{itemize}
\item \textbf{Extended histograms:}
  Ranking query plans that include imputation requires estimates of the distribution of missing values in the base tables and in the outputs of sub-plans.
  We extend standard histograms to include the count of missing values for each attribute.
  On the base tables, this provides an exact count of the missing values.
  In sub-plans we use simple heuristics---as discussed in Section~\ref{sec:cardinal}---to estimate the number of missing values after applying both standard and imputation operators.
  These cardinality estimates are critical to the optimizer's ability to compare the performance and imputation quality of different plans.

\item \textbf{Dirty sets:}
  The planner needs to know which attributes in the output of a plan might contain missing values so it can insert the correct imputation operators.
  To provide the planner with this information, we over-approximate the set of attributes in a plan that may have missing values.
  Each plan has an associated \emph{dirty set}, described in Section~\ref{sec:placement}, which tracks these attributes.

\item \textbf{Imputation operators:}
  We extend the set of logical operations available to the planner with the \emph{Impute} and \emph{Drop} operators.
  The database must have implementations for both operators and be able to correctly place them while planning.
  In addition to the normal heuristics for estimating query runtime, these operators have a cost function $\textsc{Penalty}$ (Section~\ref{sec:quality}) which estimates the quality of their output.
  The planner must be able to optimize both cost functions and select an appropriate plan from the set of Pareto-optimal plans.
  
  % While the \emph{Impute} and \emph{Drop} operations are not exposed directly to the user, they are a core extension the the standard logical operator nodes in the planner.
  % Both operate on a subset of attributes, determined by a combination of the dirty set and planner-provided attributes. These are used to enumerate local imputation alternatives.
  % Both operators can compute cardinality and adjust histogram estimates.
  % In contrast to standard operators, they also provide $\textsc{Penalty}$ and $\textsc{Time}$ estimates (\Cref{sec:cost-model}), based on the underlying imputation algorithm and the data input.

\item \textbf{Integrated optimizer:}
  We believe that it would be relatively simple to extend an existing query optimizer to insert imputations immediately after scanning the base tables.
  However, integrating the imputation placement rules into the optimizer allows us to explore a large space of query plans which contain imputations.
  In particular, this tight integration allows us to jointly choose the most effective join order and imputation placement.
\end{itemize}
}

%%% Local Variables:
%%% mode: latex
%%% TeX-master: "main"
%%% End:

\newcommand{\demorows}{10175}
\newcommand{\labexrows}{9813}

\section{Experiments}\label{sec:experiments}
To evaluate the performance of our approach,  we implemented
a prototype system in Java, following a traditional iterator model.
For our experiments we plan and execute queries
for two separate survey-based data sets, showing that our system
is well suited for early dataset exploration.

\subsection{Data sets} \label{subsec:datasets}
We collected three data sets our experiments.
For all data sets, we selected a subset of the original attributes.
We also enumerated strings and encoded them with an appropriate integer value.
\todo{describe all data transformations. jose?}

\subsubsection{CDC NHANES}
For our first set of experiments, we use survey data collected by the 
Centers for Disease Control and Prevention (CDC) in the United States. We
experiment on a set of tables collected as part of the 2013--2014 National
Health and Nutrition Examination Survey (NHANES), a series of studies
conducted by the CDC on a national sample of several thousand individuals~\cite{cdc-data}.
The data consists on survey responses, physical examinations, and laboratory
results, amongst others.

There are 6 tables in the NHANES data set. We use three tables for our experiments:

\begin{itemize}
	\item \emph{Demographics}: demographic information of subjects
	\item \emph{Examinations}: physical exam results
	\item \emph{Laboratory}: laboratory exam results
\end{itemize}

The original tables have a large number of attributes, in some cases providing more granular tests results or alternative metrics.
We focused on a subset of the attributes for each table to simplify the presentation and exploration of queries.
\Cref{table:nhanes-description} shows the attributes selected, along with the percentage of null values for each attribute.
For readability, we have replaced the NHANES variable names with self-explanatory attribute names.

\begin{table}
  \centering
  \begin{subtable}{0.5\textwidth}
    \centering
    \begin{tabular}{lS[table-format=2.2]}
\toprule
\textbf{Attribute} &  \textbf{Missing} \\
\midrule
age\_months &      93.39\ \% \\
age\_yrs &       0.00\ \% \\
gender &       0.00\ \% \\
id &       0.00\ \% \\
income &       1.31\ \% \\
is\_citizen &       0.04\ \% \\
marital\_status &      43.30\ \% \\
num\_people\_household &       0.00\ \% \\
time\_in\_us &      81.25\ \% \\
years\_edu\_children &      72.45\ \% \\
\bottomrule
\end{tabular}

%%% Local Variables:
%%% mode: latex
%%% TeX-master: "../main"
%%% End:

    \caption{Demographics. \demorows{} rows.}
  \end{subtable}
  \par\medskip
  \begin{subtable}{0.5\textwidth}
    \centering
    \begin{tabular}{llr}
\toprule
\textbf{Attribute} &  \textbf{\% Missing} \\
\midrule
albumin &      17.95 \\
blood\_lead &      46.86 \\
blood\_selenium &      46.86 \\
cholesterol &      22.31 \\
creatine &      72.59 \\
hematocrit &      12.93 \\
id &       0.00 \\
triglyceride &      67.94 \\
vitamin\_b12 &      45.83 \\
white\_blood\_cell\_ct &      12.93 \\
\bottomrule
\end{tabular}

    \caption{Laboratory Results. \labexrows{} rows.}
  \end{subtable}
  \par\medskip  
  \begin{subtable}{0.5\textwidth}
    \centering
    \begin{tabular}{llr}
\toprule
 \textbf{Table} &                \textbf{Attribute} &  \textbf{\% Missing} \\
\midrule
 exams &        arm\_circumference &       5.22 \\
 exams &      blood\_pressure\_secs &       3.11 \\
 exams &  blood\_pressure\_systolic &      26.91 \\
 exams &          body\_mass\_index &       7.72 \\
 exams &                cuff\_size &      23.14 \\
 exams &       head\_circumference &      97.67 \\
 exams &                   height &       7.60 \\
 exams &                       id &       0.00 \\
 exams &      waist\_circumference &      11.74 \\
 exams &                   weight &       0.92 \\
\bottomrule
\end{tabular}

    \caption{Physical Results. \labexrows{} rows.}
  \end{subtable}
  \par\medskip  
  \caption{Missing value distribution for each table/attribute in CDC NHANES 2013--2014 data.}\label{table:nhanes-description} 
\end{table}

\subsubsection{freeCodeCamp 2016 New Coder Survey}
For our second set of experiments, we chose to use data collected
by freeCodeCamp as a part of a survey of new software developers
(both professional and amateur)~\cite{fcc-data}. freeCodeCamp is an open-source
community that helps users learn how to program. Their \textit{2016 New Coder Survey} consists of responses by over 15,000 people to 48 different
demographic and programming-related questions.
The survey targeted users who were related to coding organizations.

We use a version of the data that has been pre-processed, but where missing values remain.
\todobox{Give example of expected missing value}{However, many of the missing values are expected because the data has been de-normalized.}
The original dataset has 15,620 rows and 113 attributes.
We chose a subset of 17 attributes, shown in~\Cref{table:fcc-description}, along with the percentage of missing values.

\begin{table}
  \centering
  \begin{tabular}{lr}
\toprule
            \textbf{Attribute} &  \textbf{\% Missing} \\
\midrule
age &      12.85 \\
attendedbootcamp &       1.54 \\
 bootcampfinish &      94.03 \\
 bootcampfulljobafter &      95.93 \\
 bootcamploanyesno &      94.02 \\
 bootcamppostsalary &      97.89 \\
childrennumber &      83.65 \\
citypopulation &      12.74 \\
commutetime &      46.61 \\
countrycitizen &      12.59 \\
gender &      12.00 \\
hourslearning &       4.34 \\
income &      53.08 \\
moneyforlearning &       6.02 \\
monthsprogramming &       3.88 \\
schooldegree &      12.43 \\
studentdebtowe &      77.50 \\
\bottomrule
\end{tabular}

  \caption{Missing values in freeCodeCamp Survey Data}\label{table:fcc-description} 
\end{table}

\subsubsection{American Community Survey}
For our final experiment, we introduce a simple aggregate query over data from the American Community Survey (ACS), which
provides a number of public data sets collected by the U.S. Census Bureau.
We used a cleaned version of the 2012 Public Use Microdata Sample (PUMS) data kindly provided by the authors of~\cite{akande2015empirical}.
Given that the data had been cleaned, we artificially dirtied it by removing 40\% of the values uniformly at random.
The final dataset consists of 671,153 rows and 37 integer columns.

\subsection{Queries}
We collected a set of queries (\Cref{fig:queries}) that we think are interesting to plan.
We believe that they could reasonably be written by a user in the course of data analysis.

The queries on the CDC NHANES data consist not only of projections and selections, but also interesting joins and aggregates.
Our aim was to craft meaningful queries
that would provide performance figures relevant to practitioners using
similar datasets.

Queries 1-4 are on the CDC data. In~\Cref{q1}, we calculate the average height for
users based on their income data. In~\Cref{q2}, we compare cholesterol levels for
individuals with low, medium, and high incomes spectrum and above a certain weight. In~\Cref{q3}, 
we extract the maximum blood lead levels for children under the age of 6
years. In~\Cref{q4}, we calculate the average systolic blood pressure, by gender, for
subjects with a body mass index indicating obesity. 

Queries 5-8 are on the freeCodeCamp data.
\Cref{q5} calculates the average income for survey participants, based on their bootcamp attendance.
\Cref{q6} estimates the average age of women from the United States who participated.
\Cref{q7} calculates the average amount of money survey participants with student debt spent on learning based on their school degree.
\Cref{q8} joins the freeCodeCamp data with a reference table provided by the World Bank which summarizes GDP per-capita across various countries~\cite{worldbank-data}.
The query calculates the average GDP per-capita of countries with and without bootcamp participants. 

\begin{table*}
  \todo{do we really need headers on these tables?}
  \centering
  \begin{subtable}{\linewidth}
    \newcounter{queryno}
\begin{tabular}{cl}
\toprule
\# & \multicolumn{1}{c}{Query} \\
\midrule
1 & 
\begin{minipage}{6in}
\begin{lstlisting}[breaklines]
SELECT income, AVG(height)
FROM demo, exams
WHERE demo.id = exams.id
GROUP BY income;
\end{lstlisting}
\end{minipage}\refstepcounter{queryno} \label{q1} \\
2 & 
\begin{minipage}{6in}
\begin{lstlisting}[breaklines]
SELECT income, AVG(cholesterol)
FROM demo, exams, labs
WHERE demo.id = exams.id AND exams.id = labs.id AND
      income >= 13 AND income <= 15 AND weight >= 63
GROUP BY income;
\end{lstlisting}
\end{minipage}
\refstepcounter{queryno} \label{q2} \\
3 & 
\begin{minipage}{6in}
\begin{lstlisting}[breaklines]
SELECT MAX(blood_lead)
FROM demo, exams, labs
WHERE demo.id = labs.id AND labs.id = exams.id AND age_yrs <= 6;
\end{lstlisting}
\end{minipage}\refstepcounter{queryno} \label{q3}\\
4 & 
\begin{minipage}{6in}
\begin{lstlisting}[breaklines]
SELECT gender, AVG(blood_pressure_systolic)
FROM demo, labs, exams
WHERE demo.id = labs.id AND labs.id = exams.id AND
      body_mass_index >= 30
GROUP BY gender;
\end{lstlisting}
\end{minipage}\refstepcounter{queryno} \label{q4}\\
%5 & 
%\begin{minipage}{6in}
%\begin{lstlisting}[breaklines]
%SELECT age_yrs, gender, triglyceride, waist_circumference
%FROM demo, labs, exams
%WHERE demo.id = exams.id AND labs.id = exams.id AND
%      labs.triglyceride > 200;
%\end{lstlisting}
%\end{minipage}\refstepcounter{queryno} \label{q5}\\
\bottomrule
\end{tabular}

    \caption{Queries on CDC data}\label{fig:queries-cdc}
  \end{subtable}
  \par\medskip
  \begin{subtable}{\linewidth}
    \begin{tabular}{cl}
\toprule
\# & \multicolumn{1}{c}{Query} \\
\midrule
6 & 
\begin{minipage}{6in}
\begin{lstlisting}[breaklines]
SELECT attendedbootcamp, AVG(income)
FROM fcc
WHERE income >= 50000
GROUP BY attendedbootcamp;
\end{lstlisting}
\end{minipage}\refstepcounter{queryno} \label[query]{q6} \\
7 & 
\begin{minipage}{6in}
\begin{lstlisting}[breaklines]
SELECT AVG(commutetime)
FROM fcc
WHERE gender = "female" AND countrycitizen = "United States";
\end{lstlisting}
\end{minipage}\refstepcounter{queryno} \label[query]{q7} \\
8 & 
\begin{minipage}{6in}
\begin{lstlisting}[breaklines]
SELECT schooldegree, AVG(studentdebtowe)
FROM fcc
WHERE studentdebtowe > 0 AND schooldegree >= 0
GROUP BY schooldegree;
\end{lstlisting}
\end{minipage}\refstepcounter{queryno} \label[query]{q8}\\
9 & 
\begin{minipage}{6in}
\begin{lstlisting}[breaklines]
SELECT attendedbootcamp, AVG(gdp_per_capita)
FROM fcc, gdp
WHERE fcc.countrycitizen = gdp.country AND age >= 18
GROUP BY attendedbootcamp;
\end{lstlisting}
\end{minipage}\refstepcounter{queryno} \label[query]{q9}\\
\bottomrule
\end{tabular}

    \caption{Queries on freeCodeCamp data}\label{fig:queries-fcc}
  \end{subtable}
  \par\medskip  
  \caption{Queries used in our experiments.}\label{fig:queries}
\end{table*}

%\begin{table*}
%  \centerfloat
%  \begin{tabular}{cSSSSSSS}
\toprule
\multicolumn{2}{c}{} & \multicolumn{2}{c}{Imputed ($\alpha=0.0$)} & \multicolumn{2}{c}{Imputed ($\alpha=1.0$)} \\
\cmidrule(r){3-4}
\cmidrule(l){5-6}
\# & \multicolumn{1}{c}{Base error} & \multicolumn{1}{c}{Error} & \multicolumn{1}{c}{Time (s)} & \multicolumn{1}{c}{Error} & \multicolumn{1}{c}{Time (s)} \\
\midrule
\ref{q1} & \multicolumn{1}{c}{--} & \multicolumn{1}{c}{--} & 1 & \multicolumn{1}{c}{--} & 17 \\
\ref{q2} & 1.66e+04 & 0.00e+00 & 1 & -1.14e+04 & 9 \\
\ref{q3} & 4.78e+04 & 0.00e+00 & 3 & -4.77e+04 & 9 \\
\ref{q4} & 1.86e-03 & 0.00e+00 & 2 & 5.07e-02 & 32 \\
\ref{q5} & 2.52e+05 & 0.00e+00 & 1 & -2.19e+05 & 25 \\
\ref{q6} & 9.89e-04 & 0.00e+00 & 1 & 9.89e-02 & 8 \\
\ref{q7} & 3.76e-02 & 0.00e+00 & 2 & 4.21e-02 & 55 \\
\ref{q8} & 0.00e+00 & 0.00e+00 & 2 & 0.00e+00 & 3 \\
\ref{q9} & \multicolumn{1}{c}{--} & \multicolumn{1}{c}{--} & 0 & \multicolumn{1}{c}{--} & 7 \\
\ref{q10} & \multicolumn{1}{c}{--} & \multicolumn{1}{c}{--} & 0 & \multicolumn{1}{c}{--} & 0 \\
\ref{q11} & 0.00e+00 & 0.00e+00 & 0 & 1.00e-01 & 0 \\
\ref{q12} & \multicolumn{1}{c}{--} & \multicolumn{1}{c}{--} & 0 & \multicolumn{1}{c}{--} & 0 \\
\ref{q13} & 0.00e+00 & 0.00e+00 & 0 & 3.81e+01 & 72 \\
\bottomrule
\end{tabular}

%    \caption{Base error, percent change in error and and running time for queries
%    with different imputation levels. Base error is the root-mean-square error (RMSE) between the query run on clean
%    data and the query run on dirty data without imputation. Change in error is relative to the base error.}
%  \label{fig:experiments}
%\end{table*}

All experiments were run on a single Amazon Web Services EC2 {\tt c4.xlarge} instance, with
four 2.9 GHz Intel Xeon E5--2666 v3 virtual CPUs and 7.5 GiB of main memory, on Debian Linux.

\Cref{fig:runtimes} shows a summary of the performance results. The first plot
shows average running times for \ProjectName{} plans in two configurations:
quality-optimized and runtime-optimized. The second plot shows, for comparison, the average
running times required for each query when performing imputation at the base tables.  In all
cases, running a query through \ProjectName{} is cheaper than imputing a single base table,
with performance improvement on a factor of ten\todo{give range of performance}. This performance differential means it is feasible
to explore multiple imputations operations (including more expensive operators) when using
\ProjectName{}, in contrast to the traditional approach of base table imputation.

\begin{figure}
  \todo{change this to a single graph of X-times faster than base approach}
  \begin{subfigure}{\linewidth}
    \includegraphics[width=\columnwidth]{figures/running_times_imputedb.png}
    \caption{Query runtimes with \ProjectName{}}\label{subfig:project-runtime-queries}
  \end{subfigure}
  ~
  \begin{subfigure}{\linewidth}
    \includegraphics[width=\columnwidth]{figures/running_times_base_tables.png}
    \caption{Query runtimes with imputation on base tables}\label{subfig:base-runtime-queries}
  \end{subfigure}
  \par\medskip
  \caption{Comparing \ProjectName{} runtimes for each query
    (\Cref{subfig:project-runtime-queries}) with the runtime for a query when base-table
    imputations are used (\Cref{subfig:base-runtime-queries})}\label{fig:runtimes}
\end{figure}

%\Cref{fig:plantimes} provides a summary of the planning times for each of the queries.
%We exclude the planning time for queries that impute at base table, as that requires no
%planning. 

In the median query across a variety of queries and choices of $\alpha$, planning
constituted 8.5 percent of total runtime. In all cases, the optimizer
returned a query plan within 40ms, with times roughly constant between levels of $\alpha$.

% The one-standard-deviation
%intervals around the mean planning time often overlap, suggesting the planning component is
%constant in $\alpha$.

%\begin{figure}
%\includegraphics[width=\columnwidth]{figures/planning_times_imputedb.png}
%\caption{Planning times for each query}
%\label{fig:plantimes}
%\end{figure}

\Cref{table:smape} shows the Symmetric-Mean-Absolute-Percentage-Error (SMAPE) for \ProjectName{}'s query results when compared to running imputation on the base tables and executing the query on the cleaned data \todo{SMAPE citation}.
Each query ran for 200 iterations \todo{iterations? is this for the imputation or is it the number of query re-runs?} in both settings and then results from the two approaches were paired up and compared tuple-wise.
We average tuple-wise absolute percentage deviations within each iteration of a query, and we report this value averaged over all iterations.
We can see that optimizing for quality indeed reduces the SMAPE of query results.
In general, the SMAPE relative to the base-imputation approach are low in all cases\todo{give actual value}, indicating that on-the-fly imputation produces similar results to imputation at the base tables.

\begin{table}
\todo{Consider switching to graph, or changing table layout..not very intuitive right now.}
\centering
\begin{tabular}{rrr}
\toprule
       &  \multicolumn{2}{c}{SMAPE} \\
 Query &  $\alpha = 0$ &  $\alpha = 1$ \\
\midrule
     1 &  0.15         &  2.30 \\
     2 &  2.65         &  2.41 \\
     3 &  0.49         &  2.14 \\
     4 &  0.31         &  0.30 \\
     5 &  1.99         &  0.54 \\
     6 &  0.30         &  0.57 \\
     7 &  6.34         &  23.87 \\
     8 &  0.26         &  0.34 \\
\bottomrule
\end{tabular}

\caption{Symmetric-Mean-Absolute-Percentage-Error for queries run under different $\alpha$
    parameterizations relative to results when imputing on base table. Values of $0.0$,
    $100.0$, or $NaN$ indicate uninformative values\todo{use one flag value (-) or explain why value is uninformative}.} %TODO MJS
\label{table:smape}
\end{table}

In many real-world cases, applying the imputation step at the base table is prohibitively expensive.
To illustrate the increasing difficulty of such an approach as datasets scale, we ran the following query over the ACS dataset:
\begin{lstlisting}
SELECT AVG(c0) FROM acs_dirty;
\end{lstlisting}
Imputing the base table and then running the query takes 75 minutes.
In contrast, \ProjectName{} executes a quality-optimized query plan in 7 seconds and a runtime-optimized plan in 1 second.
This highlights the benefit of using our system for early data exploration.
\end{figure}


%%% Local Variables:
%%% mode: latex
%%% TeX-master: "main"
%%% End:

\section{Related Work}

\subsection{Missing Values and Statistics}

Imputation of missing values is a widely studied field within the statistics and machine
learning communities. As highlighted in~\cite{gelman2006data}, missing data
can appear for a variety of reasons, including both random and conditioned on
existing values (observed and missing). Methods in the statistical community
focus on correctly modeling relationships between the attributes to factor in
varied forms of missingness. For example, Burgette and Reiter~\cite{burgette2010multiple} discuss the usage of sequential regression trees
for imputing missing data.

In~\cite{akande2015empirical}, Akande et al analyze the performance of various
multiple imputation techniques on the American Community Survey dataset. 
The computational difficulties of imputing on large base
tables are well known and can limit approaches. For example, Akande finds that
one approach (MI-GLM) is prohibitively expensive when attempting to impute on data that
includes variables with potentially large domains (ten categories in their case).
In contrast, \ProjectName{} allows users to specify a tradeoff between
information lost and runtime performance, allowing users to perform queries
directly on the entire dataset. Furthermore,
the query planner's imputation is guided by the requirements of each specific
query's operators, rather than requiring broad assumptions about query
workloads.  

\subsection{Missing Values and Databases}
There is a long history in the database community surrounding the
treatment of nulls. As early as 1973,~\cite{codd1973understanding}
provides a treatment of the semantics of null. Multiple
papers have described various (at times conflicting) treatments
of nulls~\cite{grant1977null}. \ProjectName's main design invariant - no relational operator
sees missing values for attributes it must operate on and users should never see
missing data - eliminates
the need to handle null value semantics, while guaranteeing soundness (modulo
imputation strategy) of the query evaluation.

Database system developers and others have worked on techniques to automatically
detect dirty values, whether missing or otherwise, and rectify the errors if
possible. A survey of methods and systems is provided in
\cite{hellerstein2008quantitative}.

BayesDB \cite{mansinghka2015bayesdb} provides users with a simple interface to 
leverage statistical inference techniques in a database. Non-experts
can use a simple declarative language (an extension of SQL), to specify models
which allow missing value imputation, amongst other broader functionality.
Experts can further customize strategies and express domain knowledge to
improve performance and accuracy.

While BayesDB can be used for value imputation, this step is not framed
within the context of query planning, but rather as an explicit statistical
inference step within the query language, using the \verb|INFER| operation. 

BayesDB provides a great alternative for bridging the gap between
traditional databases and sophisticated modeling software. \ProjectName{}, in
contrast, aims to remain squarely in the database realm, while allowing
users to directly express queries on a potentially larger subset of their data.

\ProjectName's cost-based query planner 
is partially based on the seminal work developed for System R's query planning\cite{blasgen1981system}.
However, in contrast to System R, \ProjectName{} performs additional histogram transformations to account
for the changing nature of missing values.

\subsection{Forecasting and Databases}
Parisi et al\cite{parisi2011embedding} introduce the idea of incorporating time-series forecast operators into
databases, along with the necessary relational algebra extensions. Their work explores the theoretical
properties of forecast operators and generalizes them into a family of operators, distinguished by
the type of predictions returned. They highlight the use of forecasting for replacing missing values.
In their follow on work\cite{parisi2013temporal}, they identify various equivalence and containment
relationships when using forecast operators, which could be used to perform query plan transformations that guarantee the same result. They
explore forecast-first and forecast-last plans, which perform forecasting operations before and after traditional
relational operators, respectively.

In contrast to this work, \ProjectName{} is targeted at generic value imputation, not necessarily tailored to 
time-series. The optimizer is not based on equivalence transformations, nor are there guarantees of equal
results under different conditions. Instead, the optimizer allows users to pick their tradeoff between
runtime cost and information lost. The search space considered by our optimizer is broader, not just
forecast-first/forecast-last plans, but rather imputation operators can be placed anywhere in the query plan
(with some restrictions). The novelty of our contribution lies in the successful incorporation of
imputation operations in non-trivial query plans with cost-based optimization.

\cite{duan2007processing} describes the Fa system and its declarative language for time-series forecasting. Their
system automatically searches the space of attribute combinations/transformations and statistical models
to produce forecasts within a given accuracy threshold. Accuracy estimates are determined using
standard techniques, such as k-fold cross-validation. 

Similarly to Fa, \ProjectName{} provides a declarative language, as
a subset of standard SQL. \ProjectName{}, however, is not searching the space of possible
imputation models, but rather the space of query plans that incorporate imputation operators. Another major point
of distinction between \ProjectName{} and Fa is that the latter doesn't consider tradeoffs between accuracy and computation time, but rather returns the most accurate forecast (with some stopping criterion).



%%% Local Variables:
%%% mode: latex
%%% TeX-master: "main"
%%% End:

\section{Conclusion}

As shown in \ProjectName{}, missing values and their imputation can successfully be integrated into the relational calculus and
existing plan optimization frameworks. We implement imputation actions, such as dropping or imputing values with a machine
learning technique, as operators in the algebra and use a simple, but effective, cost model to consider tradeoffs in 
information loss and time. In effect, user preferences for one or the other can be easily incorporated by adjusting a parameter.
Simple histogram transformations
provide incrementally updated cardinality estimates across operators in the plans, which allow us to provide more accurate cost estimates.
By taking a dynamic programming approach, we can consider a variety of operator placements
and input columns, while keeping planning tractable in real-world examples.

In our experiments, we considered a series of analytic queries (using relevant aggregates) and set queries (with simple projections) on real-world
American Community Survey data and synthetic data
and showed that the chosen query plan improves accuracy of results relative to simply ignoring missing values. Furthermore,
the variety of imputation operator placements across queries further emphasizes the lack of flexibility imposed by the coarse-grained
pre-processing approach to imputation. By allowing different imputation strategies for different queries, we
take a fine-grained approach to missing data, better addressing the often differing needs of database users.

We highlight the long history of dealing with missing data both in the statistical learning and database communities.
Similarly to existing work, we consider the impact of null values in databases and develop a simple set of invariants to 
successfully plan around them. In contrast to the statistical learning work, our emphasis is not on the specific algorithm
used to impute but rather on the timing of imputation in the execution of a query. In contrast to existing database work,
we incorporate imputation into a cost-based optimizer and hide any details
regarding missing values inside the system, allowing users to use traditional SQL and engage in normal workloads.

\subsection{Future Work}
\ProjectName{} opens up multiple avenues for further work. For instance, we
could extend our minimal/maximal imputation operators to consider global
information, such as the specific columns needed in operators higher up in the
query plan.  Our optimizer uses the same machine learning algorithm in all
instances of imputation operators. Integrating multiple possible algorithms
could allow for further fine-grained imputation, and honing the time complexity and
loss of these algorithms for an iterator model database could facilitate more
intelligent query plans and consistent interpretations of $\alpha$ across
strategies. For example, algorithms that learn in an online manner could
increase the efficiency of the system.  Furthermore, a multiple imputation
strategy could be followed for queries involving certain aggregates. Finally,
the underlying database for \ProjectName{} is far from a full-featured,
production database, so a natural step is to integrate imputation planning in a
widely-adopted production-quality database. This will likely expose further
opportunities for improvement.





%%% Local Variables:
%%% mode: latex
%%% TeX-master: "main"
%%% End:


\section{Acknowledgments}
The authors would like to acknowledge \AtNextCite{\defcounter{maxnames}{3}}\citeauthor{akande2015empirical} for giving us a cleaned copy of the 2012 ACS PUMS.

\bibstyle{abbrv}
\printbibliography
\balance

\end{document}

%%% Local Variables:
%%% mode: latex
%%% TeX-master: t
%%% End:
