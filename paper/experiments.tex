\section{Experiments}

To evaluate the performance of our system, we generated plans for queries on two data sets: the American Community Survey and a synthetic data set containing business information.\todo{Synthetic queries?}

\subsection{Data sets}

The American Community Survey provides a number of public data sets collected by the U.S. Census Bureau.
We used a cleaned version of the 2012 Public Use Microdata Sample (PUMS) data kindly provided by the authors of~\cite{akande2015empirical}.
The cleaning procedure is described in detail in~\cite{akande2015empirical}, but to summarize, the following data were removed:
\begin{itemize}
\item Rows corresponding to vacant houses or single occupant households.
\item Identification variables (e.g. area codes, serial numbers).
\item Flag variables.
\item Continuous variables.
\end{itemize}
The final data set consists of a single table with 671,153 rows and 35 columns, where all column entries are integers.

\todo{Describe synthetic data}

\subsection{Queries}

We collected a set of queries that we think are both representative, in that they could reasonably be written by a user in the course of data analysis, and interesting to plan.

The queries on the ACS data all consist only of selections, filters, and aggregations.
The ACS data is contained in a single table, so there are relatively few join queries that are interesting on this data set.
However, even with select, filter, and aggregate, there are interesting choices to make with imputation placement.

\begin{figure}
  \centerfloat
  \begin{tabular}{ll}
\toprule
\# & Query \\
\midrule
0 & \verb|SELECT BLD FROM acs;| \label{q0} \\
1 & \verb|SELECT BLD as units_in_structure, COUNT(ST) as estimate FROM acs GROUP BY BLD;| \label{q1} \\
2 & \verb|SELECT BATH as has_bath, COUNT(ST) as ct FROM acs GROUP BY BATH;| \label{q2} \\
3 & \verb|SELECT ACR as lotsize, AVG(BDSP) as avg_num_bedrooms FROM acs GROUP BY ACR;| \label{q3} \\
4 & \verb|SELECT PSF as has_sub_families, SUM(NP) as num_people FROM acs GROUP BY PSF;| \label{q4} \\
5 & \verb|SELECT AVG(NP) as avg_num_people FROM acs;| \label{q5} \\
6 & \verb|SELECT BDSP as num_bedrooms, AVG(ACR) as avg_lot_size FROM acs WHERE VEH >= 2 GROUP BY BDSP;| \label{q6} \\
\bottomrule
\end{tabular}

  \caption{The queries used in our experiments.}
\end{figure}

\begin{figure}
  \centerfloat
  \begin{tabular}{cSSSSSSS}
\toprule
\multicolumn{2}{c}{} & \multicolumn{2}{c}{Imputed ($\alpha=0.0$)} & \multicolumn{2}{c}{Imputed ($\alpha=1.0$)} \\
\cmidrule(r){3-4}
\cmidrule(l){5-6}
\# & \multicolumn{1}{c}{Base error} & \multicolumn{1}{c}{Error} & \multicolumn{1}{c}{Time (s)} & \multicolumn{1}{c}{Error} & \multicolumn{1}{c}{Time (s)} \\
\midrule
\ref{q1} & \multicolumn{1}{c}{--} & \multicolumn{1}{c}{--} & 1 & \multicolumn{1}{c}{--} & 17 \\
\ref{q2} & 1.66e+04 & 0.00e+00 & 1 & -1.14e+04 & 9 \\
\ref{q3} & 4.78e+04 & 0.00e+00 & 3 & -4.77e+04 & 9 \\
\ref{q4} & 1.86e-03 & 0.00e+00 & 2 & 5.07e-02 & 32 \\
\ref{q5} & 2.52e+05 & 0.00e+00 & 1 & -2.19e+05 & 25 \\
\ref{q6} & 9.89e-04 & 0.00e+00 & 1 & 9.89e-02 & 8 \\
\ref{q7} & 3.76e-02 & 0.00e+00 & 2 & 4.21e-02 & 55 \\
\ref{q8} & 0.00e+00 & 0.00e+00 & 2 & 0.00e+00 & 3 \\
\ref{q9} & \multicolumn{1}{c}{--} & \multicolumn{1}{c}{--} & 0 & \multicolumn{1}{c}{--} & 7 \\
\ref{q10} & \multicolumn{1}{c}{--} & \multicolumn{1}{c}{--} & 0 & \multicolumn{1}{c}{--} & 0 \\
\ref{q11} & 0.00e+00 & 0.00e+00 & 0 & 1.00e-01 & 0 \\
\ref{q12} & \multicolumn{1}{c}{--} & \multicolumn{1}{c}{--} & 0 & \multicolumn{1}{c}{--} & 0 \\
\ref{q13} & 0.00e+00 & 0.00e+00 & 0 & 3.81e+01 & 72 \\
\bottomrule
\end{tabular}

  \caption{Running time and error for queries with different imputation levels.}
\end{figure}

%%% Local Variables:
%%% mode: latex
%%% TeX-master: "main"
%%% End:
