\section{Implementation}
\todo{this is a preliminary version of this... clean up and trim language}
We discuss the main components required to implement an optimizer for dynamic query imputation. 
We use as reference a standard tuple-iterator prototype database.

\textbf{Extended histograms}: To plan dynamic imputation, we require estimates of the distribution of missing values in
the underlying tables. We found this information easy to store by extending our existing histograms to include the count
of missing values in each bucket across table attributes. On the base tables, this provides an exact count of the
missing fields. In intermediate planning stages, it allows us to use simple heuristics to estimate the number of missing 
values remaining after standard relational operations, such as selections. While statistics about missing value counts
could be kept separate from standard cardinality estimation histograms, cardinality estimation in the optimizer would
frequently need to adjust one based on the other, so its simplest to merge them.

\textbf{Dirty sets}: We keep track of the attributes that missing values but have not yet been imputed (using $\mu$ or $\delta$). We extend
the logical nodes in our planning data structures to carry a set of fully qualified attribute names representing the dirty set. This required 
small modifications in all standard relation operator nodes in our system.

\textbf{Additional Logical Operators}: While the $\mu$ and $\delta$ operations are not exposed directly to the user, they are a core extension
the the standard set of logical operator nodes used by a query planner. Both operators operate on a subset of attributes, determined by
a combination of the dirty set and planner-provided attributes, used to enumerate local imputation alternatives. Both operators, just like
selection or join nodes in the standard planner, can provide estimates of the new cardinality and do so by adjusting the appropriate histograms
handed off to the next operator in the plan.

\textbf{Core optimizer}: Imputations early in the plan (e.g. before a selection) could conceivably be added to existing optimizers. However, the full
planning used in \ProjectName{} is tightly integrated with core parts of the optimizer, such as join enumeration. This tight integration allows \ProjectName{}
to effectively prune out plans that are not in the relevant Pareto set. It also allows imputation costs to influence choose of join orders. For these reasons
we believe substituting the standard optimizer for a custom version tailored to dynamic imputation optimization is critical.


%%% Local Variables:
%%% mode: latex
%%% TeX-master: "main"
%%% End:
